\documentclass[8pt,aspectratio=169,hyperref={unicode=true}]{beamer}

\usefonttheme{serif}
\usepackage{fontspec}
	\setmainfont{TeX Gyre Heros}
\usepackage{unicode-math}
\usepackage{lualatex-math}
	\setmathfont{TeX Gyre Termes Math}
\usepackage{polyglossia}
\setdefaultlanguage[frenchpart=false]{french}
\setotherlanguage{english}
%\usepackage{microtype}
\usepackage[locale = FR,
            separate-uncertainty,
            multi-part-units = single,
            range-units = single]{siunitx}
	\DeclareSIUnit\an{an}
  \DeclareSIUnit{\octet}{o}
\usepackage{amsmath}
\usepackage{amsfonts}
\usepackage{amssymb}
\usepackage{array}
\usepackage{graphicx}
\graphicspath{{./Figures/}}
\usepackage{booktabs}
\usepackage{tabularx}
\usepackage{multirow}
\usepackage{multicol}
    \newcolumntype{L}{>{\raggedright\arraybackslash}X}
    \newcolumntype{R}{>{\raggedleft\arraybackslash}X}
\usepackage{makecell}
\setcellgapes{5pt}
\usepackage{xcolor}
\usepackage{tikz}
\usetikzlibrary{graphs, graphdrawing, arrows.meta} \usegdlibrary{layered, trees}
\usetikzlibrary{overlay-beamer-styles}
\usepackage{subcaption}
\usepackage[]{animate}
\usepackage{float}
\usepackage{csquotes}
\usepackage{minted}

\usetheme[secheader]{Boadilla}
\usecolortheme{seagull}
\setbeamertemplate{enumerate items}[default]
\setbeamertemplate{itemize items}{-}
\setbeamertemplate{navigation symbols}{}
\setbeamertemplate{bibliography item}{}
\setbeamerfont{framesubtitle}{size=\large}
\setbeamertemplate{section in toc}[sections numbered]
%\setbeamertemplate{subsection in toc}[subsections numbered]

\title[Déployez un modèle dans le cloud]{Projet 8 : Déployez un modèle dans le cloud}
\author[Lancelot \textsc{Leclercq}]{Lancelot \textsc{Leclercq}} 
\institute[]{}
\date[]{\small{10 juin 2022}}

\AtBeginSection[]{
  \begin{frame}
  \vfill
  \centering
    \usebeamerfont{title}\insertsectionhead\par%
  \vfill
  \end{frame}
}

\begin{document}
\setbeamercolor{background canvas}{bg=gray!20}
\begin{frame}[plain]
  \titlepage
\end{frame}

\begin{frame}{Sommaire}
  \Large
  \begin{columns}
    \begin{column}{.7\textwidth}
      \tableofcontents[hideallsubsections]
    \end{column}
  \end{columns}
\end{frame}

\section{Introduction}
\subsection{Problématique}
\begin{frame}{\insertsubsection}
  \begin{columns}
    \begin{column}{.6\textwidth}
      \begin{itemize}
        \item Préservation de la biodiversité des fruits
              \begin{itemize}
                \item traitements spécifiques pour chaque espèce de fruits par des robots cueilleurs intelligents
              \end{itemize}
        \item[]
        \item 1ère étape développer une application mobile
              \begin{itemize}
                \item Permettre à l'utilisateurs d'obtenir des informations sur un fruit à partir d'une photo
                \item[]
                \item Sensibiliser le grand public à la biodiversité des fruits
                \item[]
                \item Mettre en place une première version du moteur de classification des images de fruits
                \item[]
                \item Construire une première version de l'architecture Big Data
              \end{itemize}
      \end{itemize}

    \end{column}
    \begin{column}{.4\textwidth}
      \includegraphics[width=\textwidth]{./Logo projet big data.png}
    \end{column}
  \end{columns}
\end{frame}

\subsection{Données}
\begin{frame}{\insertsubsection}
  \begin{columns}
    \begin{column}{.6\textwidth}
      \begin{itemize}
        \item Utilisation du jeu de données Kaggle : \url{https://www.kaggle.com/datasets/moltean/fruits}
        \item[]
        \item Nombre total d'images : 90483
        \item[]
        \item Taille du jeu d'entrainement : 67692 images
        \item[]
        \item Taille du jeu de test: 22688 images
        \item[]
        \item  Nombre de fruits : 131
      \end{itemize}
    \end{column}
    \begin{column}{.4\textwidth}
      \begin{itemize}
        \item Exemple d'image de fruit :
        \item[]
      \end{itemize}
      \includegraphics[width=.8\textwidth]{./fruits-360_dataset/fruits-360/Training/Strawberry/0_100.jpg}
    \end{column}
  \end{columns}
\end{frame}

\section{Architecture Big Data}
\subsection{Spark}
\begin{frame}{\insertsubsection}
  \begin{columns}
    \begin{column}{.6\textwidth}
      \begin{itemize}
        \item
      \end{itemize}
    \end{column}
  \end{columns}
\end{frame}
\end{document}