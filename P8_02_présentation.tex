\documentclass[8pt,aspectratio=169,hyperref={unicode=true}]{beamer}

\usefonttheme{serif}
\usepackage{fontspec}
	\setmainfont{TeX Gyre Heros}
\usepackage{unicode-math}
\usepackage{lualatex-math}
	\setmathfont{TeX Gyre Termes Math}
\usepackage{polyglossia}
\setdefaultlanguage[frenchpart=false]{french}
\setotherlanguage{english}
%\usepackage{microtype}
\usepackage[locale = FR,
            separate-uncertainty,
            multi-part-units = single,
            range-units = single]{siunitx}
	\DeclareSIUnit\an{an}
  \DeclareSIUnit{\octet}{o}
\usepackage{amsmath}
\usepackage{amsfonts}
\usepackage{amssymb}
\usepackage{array}
\usepackage{graphicx}
\graphicspath{{./Figures/}}
\usepackage{booktabs}
\usepackage{tabularx}
\usepackage{multirow}
\usepackage{multicol}
    \newcolumntype{L}{>{\raggedright\arraybackslash}X}
    \newcolumntype{R}{>{\raggedleft\arraybackslash}X}
\usepackage{makecell}
\setcellgapes{5pt}
\usepackage{xcolor}
\usepackage{tikz}
\usetikzlibrary{graphs, graphdrawing, arrows.meta} \usegdlibrary{layered, trees}
\usetikzlibrary{overlay-beamer-styles}
\usepackage{subcaption}
\usepackage[]{animate}
\usepackage{float}
\usepackage{csquotes}
\usepackage{minted}

\usetheme[secheader]{Boadilla}
\usecolortheme{seagull}
\setbeamertemplate{enumerate items}[default]
\setbeamertemplate{itemize items}{-}
\setbeamertemplate{navigation symbols}{}
\setbeamertemplate{bibliography item}{}
\setbeamerfont{framesubtitle}{size=\large}
\setbeamertemplate{section in toc}[sections numbered]
%\setbeamertemplate{subsection in toc}[subsections numbered]

\title[Déployez un modèle dans le cloud]{Projet 8 : Déployez un modèle dans le cloud}
\author[Lancelot \textsc{Leclercq}]{Lancelot \textsc{Leclercq}} 
\institute[]{}
\date[]{\small{10 juin 2022}}

\AtBeginSection[]{
  \begin{frame}
  \vfill
  \centering
    \usebeamerfont{title}\insertsectionhead\par%
  \vfill
  \end{frame}
}

\begin{document}
\setbeamercolor{background canvas}{bg=gray!20}
\begin{frame}[plain]
  \titlepage
\end{frame}

\begin{frame}{Sommaire}
  \Large
  \begin{columns}
    \begin{column}{.7\textwidth}
      \tableofcontents[hideallsubsections]
    \end{column}
  \end{columns}
\end{frame}

\section{Introduction}
\subsection{Problématique}
\begin{frame}{\insertsubsection}
  \begin{columns}
    \begin{column}{.6\textwidth}
      \begin{itemize}
        \item Préservation de la biodiversité des fruits
              \begin{itemize}
                \item traitements spécifiques pour chaque espèce de fruits par des robots cueilleurs intelligents
              \end{itemize}
        \item[]
        \item 1ère étape développer une application mobile
              \begin{itemize}
                \item Permettre à l'utilisateur d'obtenir des informations sur un fruit à partir d'une photo
                \item[]
                \item Sensibiliser le grand public à la biodiversité des fruits
                \item[]
                \item Mettre en place une première version du moteur de classification des images de fruits
                \item[]
                \item Construire une première version de l'architecture Big Data
              \end{itemize}
      \end{itemize}

    \end{column}
    \begin{column}{.4\textwidth}
      \center
      \includegraphics[width=\textwidth]{./Logo projet big data.png}
    \end{column}
  \end{columns}
\end{frame}

\subsection{Données}
\begin{frame}{\insertsubsection}
  \begin{columns}
    \begin{column}{.6\textwidth}
      \begin{itemize}
        \item Utilisation du jeu de données Kaggle : \url{https://www.kaggle.com/datasets/moltean/fruits}
        \item[]
        \item Nombre total d'images : 90483
        \item[]
        \item Taille du jeu d'entrainement : 67692 images
        \item[]
        \item Taille du jeu de test : 22688 images
        \item[]
        \item  Nombre de fruits : 131
      \end{itemize}
    \end{column}
    \begin{column}{.4\textwidth}
      \begin{itemize}
        \item Exemple d'image de fruit :
        \item[]
      \end{itemize}
      \center
      \includegraphics[width=.8\textwidth]{./fruits-360_dataset/fruits-360/Training/Strawberry/0_100.jpg}
    \end{column}
  \end{columns}
\end{frame}

\section{Services AWS}
\begin{frame}{\insertsection}
  \begin{columns}
    \begin{column}{.6\textwidth}
      \begin{itemize}
        \item Utilisation d'une machine EC2 hébergée par Amazon Web Services
              \begin{itemize}
                \item Utilisation de Debian
                \item[]
                \item Installation de java, git
              \end{itemize}
        \item[]
        \item Utilisation de S3 comme espace de stockage pour la base de données d'images
      \end{itemize}
    \end{column}
    \begin{column}{.4\textwidth}
      \center
      \includegraphics[width=.8\textwidth]{./512px-Amazon_Web_Services_Logo.svg.png}
    \end{column}
  \end{columns}
\end{frame}

\section{Chaine de traitement}
\subsection{Spark}
\begin{frame}{\insertsubsection}
  \begin{columns}
    \begin{column}{.6\textwidth}
      \begin{itemize}
        \item Parallélisation des calculs
              \begin{itemize}
                \item Utilisation de plusieurs machines
                \item[]
                \item Distribution des calculs
              \end{itemize}
        \item[]
        \item Coût d'utilisation des machines à surveiller
      \end{itemize}
    \end{column}
    \begin{column}{.4\textwidth}
      \includegraphics[width=\textwidth]{./spark.pdf}
    \end{column}
  \end{columns}
  \vfill
  \begin{center}
    \tikz [rounded corners, every node/.style={anchor=west}, level sep = 4mm, >={Stealth}]
    \graph [layered layout, grow=right, nodes={draw, font=\footnotesize}, head anchor=west, tail anchor=east,
    edges=rounded corners, sibling distance=5mm]{
    Importation images ->  Détection -> "Clusterisation (1000 groupes de descripteurs)" -> "PCA (conservation de 100 composantes)" -> "création fichier .csv",
    "Détection des descripteurs (OpenCV)" [draw] // {Détection, "Clusterisation (1000 groupes de descripteurs)"},
    "Réduction de dimension"  [draw]  // {"PCA (conservation de 100 composantes)"},
    };
  \end{center}
\end{frame}

\subsection{PySpark}
\begin{frame}[fragile]{\insertsubsection}
  \begin{itemize}
    \item Utilisation de PySpark une interface Spark en python
  \end{itemize}
  \begin{columns}
    \begin{column}{.25\textwidth}
      \begin{itemize}
        \item Importation des images au format binaire :
      \end{itemize}
    \end{column}
    \begin{column}{.75\textwidth}
      \begin{minted}{python}
ImgData = spark.read.format('binaryFile') \
                  .option('pathGlobFilter', '*.jpg') \
                  .option('recursiveFileLookup', 'true') \
                  .load(path) \
                  .select('path', 'content')   
      \end{minted}
    \end{column}
  \end{columns}
  \vspace{3px}
  \hrule
  \begin{columns}
    \begin{column}{.25\textwidth}
      \begin{itemize}
        \item Création d'une colonne avec les descripteurs détectés pas OpenCV :
      \end{itemize}
    \end{column}
    \begin{column}{.75\textwidth}
      \begin{minted}{python}
udf_image = F.udf(
    get_desc,
    ArrayType(ArrayType(FloatType(), containsNull=False), containsNull=False))

ImgDesc = ImgData.withColumn("descriptors", F.explode(udf_image("content")))  
      \end{minted}
    \end{column}
  \end{columns}
  \vspace{3px}
  \hrule
  \begin{columns}
    \begin{column}{.25\textwidth}
      \begin{itemize}
        \item Réduction de dimension par PCA :
      \end{itemize}
    \end{column}
    \begin{column}{.75\textwidth}
      \begin{minted}{python}
VA = VectorAssembler(inputCols=BoVW.drop('label').columns,
                     outputCol='features')
pca = PCA(k=100, inputCol='features', outputCol='pca_features')
pipe = Pipeline(stages=[VA, pca])

pipePCA = pipe.fit(BoVW)

pcaData = pipePCA.transform(BoVW)
pcaDataDF = pcaData.select(['label', 'pca_features']).toPandas()
      \end{minted}
    \end{column}
  \end{columns}
\end{frame}

\section{Axes d'amélioration}
\begin{frame}{\insertsection}
  \begin{itemize}
    \item KMeans instable, difficile à débugger :
    \item[$\Rightarrow$] Essayer d'autres alternatives comme dask
    \item[]
    \item Utilisation d'un EMR permettant d'avoir les calculs réalisés par plusieurs machines
    \item[]
    \item Ajout d'un modèle de classification afin de classer les fruits
    \item[]
    \item Affiner les catégories avec des maturités différentes afin de cueillir les fruits au meilleur moment 
  \end{itemize}
\end{frame}
\end{document}


